% options:
% thesis=B bachelor's thesis
% thesis=M master's thesis
% czech thesis in Czech language
% slovak thesis in Slovak language
% english thesis in English language
% hidelinks remove colour boxes around hyperlinks

\documentclass[thesis=B,czech]{FITthesis}[2012/06/26]

\usepackage[utf8]{inputenc} % LaTeX source encoded as UTF-8

\usepackage{graphicx} %graphics files inclusion
% \usepackage{amsmath} %advanced maths
% \usepackage{amssymb} %additional math symbols

%\usepackage{stdpage} % nastaveni normostrany
\usepackage{dirtree} %directory tree visualisation

\usepackage{hyphenat}

% % list of acronyms
% \usepackage[acronym,nonumberlist,toc,numberedsection=autolabel]{glossaries}
% \iflanguage{czech}{\renewcommand*{\acronymname}{Seznam pou{\v z}it{\' y}ch zkratek}}{}
% \makeglossaries

\newcommand{\tg}{\mathop{\mathrm{tg}}} %cesky tangens
\newcommand{\cotg}{\mathop{\mathrm{cotg}}} %cesky cotangens

% % % % % % % % % % % % % % % % % % % % % % % % % % % % % %
% ODTUD DAL VSE ZMENTE
% % % % % % % % % % % % % % % % % % % % % % % % % % % % % %

\department{Katedra počítačových systémů}
\title{Virtuální privátní sítě}
\authorGN{Filip} %(křestní) jméno (jména) autora
\authorFN{Kofroň} %příjmení autora
\authorWithDegrees{Bc. Filip Kofroň} %jméno autora včetně současných akademických titulů

\abstractCS
{
  Virtuální privátní sítě (VPN) se staly nepochybně jednou z nejpoužívanějších metod, jak zajistit business v geograficky rozsáhlé společnosti či zabezpečit přístup k poskytovaným službám.
  V této práci jsou popsány základy, jak VPN fungují a jak jsou implementovány v praxi. Dále se práce zabývá možnými hrozbami, kterým VPN musí čelit.
}

\abstractEN
{
  Virtual private networks (a VPN) have become undoubtebly one of the most used measures to secure a business in a geographically sparse company or to secure an access to company provided services.
  This paper describes the basics of a VPN, how they work and how they are implemented in the real. This work also considers possible threats to VPNs.
}


\placeForDeclarationOfAuthenticity{TODO}

\keywordsCS
{
  VPN
}
\keywordsEN
{
  VPN
}

\begin{document}

% \newacronym{CVUT}{{\v C}VUT}{{\v C}esk{\' e} vysok{\' e} u{\v c}en{\' i} technick{\' e} v Praze}
% \newacronym{FIT}{FIT}{Fakulta informa{\v c}n{\' i}ch technologi{\' i}}

\begin{introduction}
  \label{sec:uvod}

  Veřejné sítě skýtají mnoho potenciálních nebezpečí, jak mohou být data uživatelů a služeb zachycena neoprávněnými osobami.
  Virtuální privátní síťe (VPN) rozšiřují privátní sítě přes sdílené či veřejné sítě jakými bývá nejčasteji internet.
  Pomocí ní mohou uživatelé mezi sebou posílat a přijímat data přes sdílené či veřejné sítě tak, jakoby byli společně připojeni k privátní síti a tak využívají jejich pravidel i zabezpečení. \cite{cisco_intro}
  Data na VPN jsou oddělena tak, že pouze příjemce dat je schopen tyto data přečíst.  \cite{netgear_vpn_basics}

  Původně byl název VPN učen k popisu zabezpečeného spojení. Dnes již však VPN pokrývá i privátní síťe jakými sou např. Frame Relay, Asynchronous Transfer Mode (ATM) a Multiprotocol Label Switching (MPLS).

  VPN je obvykle vytvořeno pomocí PTP (point to point) spojení, které komunikuje jako speciální spojení, jako tunel, čí jako pouhé zašifrování provozu.

  VPN, které je spojeno pomocí internetu se svými vlastnostmi podobá WAN (wide are network). Přístup uživatele k různým zdrojům je transparentní a tudíž uživatel může s těmito zdroji pracovat jako by byli v jeho privátní síti. Samozřejmě je svou povahou omezen možnostmi a dostupností připojení k internetu. Odezva a rychlost se přímo odvíjí od vzdálenosti, kterou musí pakety urazit a overhead, který má jejich zpracování. Výkon dále ovlivňují typy VPN a protokoly, na kterých běží. \cite{microsoft_intro}

  Zásadním aspektem zabezpečení VPN je šifrouvání dat, které skrz tuto síť protékají. Privátní síťě tyto vlastnosti obvykle postrádají, a proto se ppřípadný útočník může do sítě připojit a jednoduše data na síti sledovat i zaznamenávat. \cite{netgear_vpn_basics}

\end{introduction}

\chapter{Typy}

  \section{Frame Relay}

    \subsection{TODO}

  \section{Asynchronous Transfer Mode}
  \section{IP}
  \section{Multiprotocol Label Switching}

\chapter{Zabezpečení}

\begin{conclusion}

    TODO

\end{conclusion}

\bibliographystyle{csn690}
\bibliography{biblio}

\appendix

\chapter{Seznam použitých zkratek}
% \printglossaries
\begin{description}

    \item[TODO] TODO

\end{description}

\end{document}
