% options:
% thesis=B bachelor's thesis
% thesis=M master's thesis
% czech thesis in Czech language
% slovak thesis in Slovak language
% english thesis in English language
% hidelinks remove colour boxes around hyperlinks

\documentclass[thesis=B,czech]{FITthesis}[2012/06/26]

\usepackage[utf8]{inputenc} % LaTeX source encoded as UTF-8

\usepackage{graphicx} %graphics files inclusion
% \usepackage{amsmath} %advanced maths
% \usepackage{amssymb} %additional math symbols

%\usepackage{stdpage} % nastaveni normostrany
\usepackage{dirtree} %directory tree visualisation

\usepackage{hyphenat}

% % list of acronyms
% \usepackage[acronym,nonumberlist,toc,numberedsection=autolabel]{glossaries}
% \iflanguage{czech}{\renewcommand*{\acronymname}{Seznam pou{\v z}it{\' y}ch zkratek}}{}
% \makeglossaries

\newcommand{\tg}{\mathop{\mathrm{tg}}} %cesky tangens
\newcommand{\cotg}{\mathop{\mathrm{cotg}}} %cesky cotangens

% % % % % % % % % % % % % % % % % % % % % % % % % % % % % %
% ODTUD DAL VSE ZMENTE
% % % % % % % % % % % % % % % % % % % % % % % % % % % % % %

\department{Katedra počítačových systémů}
\title{Virtuální privátní sítě}
\authorGN{Filip} %(křestní) jméno (jména) autora
\authorFN{Kofroň} %příjmení autora
\authorWithDegrees{Bc. Filip Kofroň} %jméno autora včetně současných akademických titulů

\abstractCS
{
  Virtuální privátní sítě (VPN) se staly nepochybně jednou z nejpoužívanějších metod, jak zajistit business v geograficky rozsáhlé společnosti či zabezpečit přístup k poskytovaným službám.
  V této práci jsou popsány základy, jak VPN fungují a jak jsou implementovány v praxi. Dále se práce zabývá možnými hrozbami, kterým VPN musí čelit.
}

\abstractEN
{
  Virtual private networks (a VPN) have become undoubtebly one of the most used measures to secure a business in a geographically sparse company or to secure an access to company provided services.
  This paper describes the basics of a VPN, how they work and how they are implemented in the real. This work also considers possible threats to VPNs.
}


\placeForDeclarationOfAuthenticity{TODO}

\keywordsCS
{
  VPN
}
\keywordsEN
{
  VPN
}

\begin{document}

% \newacronym{CVUT}{{\v C}VUT}{{\v C}esk{\' e} vysok{\' e} u{\v c}en{\' i} technick{\' e} v Praze}
% \newacronym{FIT}{FIT}{Fakulta informa{\v c}n{\' i}ch technologi{\' i}}

\begin{introduction}
  \label{sec:uvod}

  Veřejné sítě skýtají mnoho potenciálních nebezpečí, jak mohou být data uživatelů a služeb zachycena neoprávněnými osobami.
  Virtuální privátní síťe (VPN) rozšiřují privátní sítě přes sdílené či veřejné sítě jakými bývá nejčasteji internet.
  Pomocí ní mohou uživatelé mezi sebou posílat a přijímat data přes sdílené či veřejné sítě tak, jakoby byli společně připojeni k privátní síti a tak využívají jejich pravidel i zabezpečení. \cite{cisco_intro}
  Data na VPN jsou oddělena tak, že pouze příjemce dat je schopen tyto data přečíst.  \cite{netgear_vpn_basics}

  Původně byl název VPN učen k popisu zabezpečeného spojení. Dnes již však VPN pokrývá i privátní síťe jakými sou např. Frame Relay, Asynchronous Transfer Mode (ATM) a Multiprotocol Label Switching (MPLS).

  VPN je obvykle vytvořeno pomocí PTP (point to point) spojení, které komunikuje jako speciální spojení, jako tunel, čí jako pouhé zašifrování provozu.

  VPN, které je spojeno pomocí internetu se svými vlastnostmi podobá WAN (wide are network). Přístup uživatele k různým zdrojům je transparentní a tudíž uživatel může s těmito zdroji pracovat jako by byli v jeho privátní síti. Samozřejmě je svou povahou omezen možnostmi a dostupností připojení k internetu. Odezva a rychlost se přímo odvíjí od vzdálenosti, kterou musí pakety urazit a overhead, který má jejich zpracování. Výkon dále ovlivňují typy VPN a protokoly, na kterých běží. \cite{microsoft_intro}

  Zásadním aspektem zabezpečení VPN je šifrouvání dat, které skrz tuto síť protékají. Privátní síťě tyto vlastnosti obvykle postrádají, a proto se ppřípadný útočník může do sítě připojit a jednoduše data na síti sledovat i zaznamenávat. \cite{netgear_vpn_basics}

\end{introduction}

\chapter{Typy}

  Typy druhů VPN jsou zde rozdělené podle vrstev OSI modelu, na kterých probíhá ona virtualizace (ne co se na nich virtualizuje). U VPN se setkaváme s vrstvami 2, 2.5 (případ MPLS) a 3.

  \section{OSI vrstva 2}

	Typy VPN na OSI vrstvě 2 virtualizují pomocí rámců.

    \subsection{Frame Relay}

      \subsubsection{Popis Frame Relay}

        Přepínání rámců (Frame relay) je standardizovaná technologie pro WAN (wide area network), který specifikuje jak fyzické, tak logické linkové vrstvy digitalních telekomunikačních kanálů používající přepínání packetů (packet switching). Navržena původně pro přenost přes Integrated Services Digital Network (ISDN), může i dnes být tato technologie použita v kontextu mnoha jiných síťových rozhraních.

        Poskytovatelé připojení obvykle implementují Voice and Data Frame Relay (VoFR) jako techniku zabalení provozu. Často je tak  použito mezi lokálními síťemi (LAN) přes rozsáhlé sítě WAN. Každý koncový uživatel takto získá soukromé nebo prpnajaté spojení do uzlu Frame Relay. Frame relay síť řídí přenos přes často se měnící cestu transparentně do všech cílových masově používaných WAn protokolů. Je mnohem méně nákladnější než pronajaté linky a to je jeden z hlavních důvodů pro jeho popularitu. Extrémní jednoduchost konfigurace uživatelských zařízeních v frame relay síti nabízí další důvod pro jeho popularitu.

        S nástupem ethernetu přes optické kabely, MPLS a dedikovaných propojovacích služeb jako jsou kabelový modem či DSL je zájem o frame relay protokol a jeho zaobalování provozu čím dál méně žádané. Přesto se však stále uplatňují v oblastech kde DSL a kabelové modemy nejsou k dispozici. V takovýchto případech zůstává jako nejlevnjší metoda nevytáčeného spojení 64 kbit/s frame relay linka. \cite{frame_relay_basics}

      \subsubsection{Frame Relay jako VPN}

        Frame relay sice umožňuje provozovat VPN, ale naprosto zásadní důvod, proč se dnes již jako VPN příliš nepoužíva, je ten, že jeho provoz není šifrovaný. Veškerá bezpečnost závisí na poskytovali. Ten je vlastně dalším důvodem, proč je Frame Relay jako VPN nevhodný. Všechny připojené uzly VPN musí sdílet stejného poskytovatele Frame Relay. To je velmi nevhodné, pokud uživateli takové sítě jsou společnosti, které mají pobočky na geograficky velmi odlehlých oblastech. \cite{frame_relay_vs_vpn}

    \subsection{Asynchronous Transfer Mode}

      \subsubsection{Popis ATM}
        Asynchronous Transfer Mode, zkráceně ATM, byl v osmdesátých a devadesátých letech standard pro vysokorychlostní (155 Mbps až 622 Mbps) síťovou architekturu. Zabezpečuje Quality of Service (QoS) pro přenos hlasu a videa. Dříve byl označován jako telefonie „další generace“ (technické kořeny v telefonním světě). Umožňuje přenos IP datagramů. Pracuje s přepojováním paketů (pakety pevné délky, zvané buňky, anglicky cell) užitím virtuálních okruhů.

        Je to síťový protokol přenosu dat po buňkách, který rozděluje přenos dat na malé kousky (buňky) s pevnou délkou (53 bytů; 48 bytů dat a 5 bytů záhlaví) místo paketů, které se délkou liší (užívané v LAN). Je to technologie orientovaná na spojení, ve které spojení je vytvořeno mezi dvěma koncovými body ještě předtím, než začne výměna dat.\cite{atm_wiki}

      \subsubsection{VPN pomocí ATM}

        VPN přes ATM má stejné nevýhody jako u VPN přes Frame relaz. Stejne jako Frame Relay ATM pouze zaobaluje určitou komunikaci.
        \cite{atm_vpn}

  \section{OSI vrstva 2.5}

    \subsection{Multiprotocol Label Switching}

      \subsubsection{Popis}
	  
		  Multiprotokolové přepojování podle návěští (anglicky Multiprotocol Label Switching, MPLS) je mechanismus směrování síťového provozu používaný ve vysokorychlostních telekomunikačních sítích, který pro směrování dat mezi síťovými uzly nepoužívá relativně dlouhé a protokolově závislé síťové adresy, ale krátká návěští pevné délky.
		  Tím se obchází prohledávání směrovacích tabulek. Návěští neidentifikují koncové body, ale virtuální spoje (cesty) mezi vzdálenými uzly. MPLS může zapouzdřovat pakety různých síťových protokolů a pro přenos používat nejrůznější technologie včetně linek T1 a E1, ATM, Frame Relay a DSL.

      \subsubsection{MPLS a VPN}

		MPLS VPN je rodina metod pro využití výkonu multiprotokolové přepojování podle návěští (MPLS) k vytvoření VPN.
		MPLS VPN dává síťovým správcům flexibilitu transportovat a routovat různé typy síťového provozu použitím technologií MPLS.

		V praxi se pužívají 3 typy MPLS VPN:
		
		\begin{itemize}
		  \item Point-to-point (Pseudowire)
		  \item Layer 2 (VPLS)
		  \item Layer 3 (VPRN)
		\end{itemize}
		
		Point-to-Point MPLS VPN používají VLL (virtual leased lines - virtuální pronajaté linky) k poskytování point-to-point konektivity 2. úrovně mezi dvěma oblastmi. Ethernet, TDM a ATM rámce mohou být přenášeny (zaobaleny) v těchto VLL linkách.
		
		MPLS VPN úrovně 2 (Layer 2) nebo VPLS (virtuální privátní LAN služba - virtual private LAN service) nabízí "přepínač v cloudu". VPLS poskytuje možnost zozšířit VLAN mezi více vzdálenými oblastmi. VPN druhé úrovně jsou typicky použity k routování zvuku, videa a AMI provozu mezi stanicemi a datovými centry.
		
		MPLS VPN úrovně 3 nebo VPRN (virtuální privátní routovaná síť - virtual private routed network), využívá VRF 3. úrovně (VPN/virtuální routování a forwarding) k segmentování routovacích tabulek pro každého zákazníka služby.
		Zákazník se spáruje s routerem poskytovatele služby a prohodí si routání, které se ukládá do speciálního zákaznického routovací tabulky. Multiprotokolové BGP (MP-BGP) je potřebné v cloudu k provozu služby, které zvyšuje komplexitu a náročnost implementace.
		LMPLS VPN úrovně 3 nejsou typicky nasazovány na sítě poskytovatelů připojení kvůli jejich komplexitě however, VPN 3. úrovně se spíše používá k routování provozu mezi korporátními oblastmi a datacentry.
		\cite{mpls_rfc}\cite{mpls_guide}
  \section{OSI vrstva 3}

    Typy VPN na OSI vrstvě 3 se lyší zejména protokoly, ale dosahují zhruba stejných výkonů. Pracují obvykle na veřejných sítích typu internet, a proto je u ních těžší zajistit kvalitu služeb (Quality of Service - QoS).

    \subsection{IPsec}

      \subsubsection{Popis}
        Internet Protocol Security (IPsec) je sada protokolů pro zabezpečení internetových protokolů (Internet Protocol - IP) sloužících ke komunikaci s autentizací a šifrováním každého IP paketu z celého provozu komunikace. IPsec obsahuje protokoly pro zprostředkování vzájemné autentizace mezi agenty na začátku sezení a výměna krzptografických klíčů, které se použijí v průběhu sezení. IPSec může být použít pro ochranu dat tekoucích mezi dvoumi hosty (host-to-host), mezi dvoumi zabezpečenými bránami (network-to-network) nebo mezi zabezpečenými bránami a hostem (network-to-host).

        Zabezpečení internetového protokolu používa zabezpečení pomocí kryptografických služeb, které slouží k ochraně komunikací přes IP síťě. IPSec podporuje autentizaci na úrovní síťe, autentizaci původu dat, zajišťuje datovou integritu, důvěrnost dat a také ochranu proti tzv. Replay útoku.

        IPSec je tzv. koncové zabespečévací schéma (end-to-end) operující ve vrstvě TCP/IP (Internet Protocol Suite), kdežto jiné používané zabezpečovací systémy internetu, kterými jsou například Transport Layer Security (TLS) a Secure Shell (SSH), operují ve vyšších vrstvách v tzv. aplikační vrstvě OSI modelu. IPsec však chrání veškerý provoz po IP síti. Tím jsou automaticky všechny aplikace zabezpečené na vrstvě IP.
        \cite{ipsec_basics}

      \subsubsection{IPsec a VPN}

        IPsec umožňuje vytvořit transparentně šifrovaný tunel mezi hosty. Poskytuje tak hlavní výhody zabezpečení VPN. \cite{ipsec_vpn}


    \subsection{OpenVPN}

      OpenVPN je velmi zajímavým VPN bežícím v uživatelském prostoru, který je jak svobodný, tak široce rozšířen. Proto se mu bude tato podsekce věnovat více do hloubky než ostatním, které buď zastarávají nebo jsou komerční varianty.

      \subsubsection{Základní popis}

        OpenVPN je svobodný software s otevřeným zdrojovým kódem, který implementuje techniky virtuální privátní síťe (VPN) pro vytváření point-to-point nebo site-to-site spojení v routované nebo přemostěné konfiguraci (bridged) a pro řízení vzdáleného přístupu. Používá svůj vlastní bezpečnostní protokol\cite{openvpn_basics_1}, který využíva SSL/TLS pro výměnu klíčů. Tento protokol dokáže traverzovat překladače síťových adres (NAT) a firewally. Prvotně byl napsán Jamesem Yonanem a je publikován pod licencí GNU General Public License (GPL).\cite{openvpn_basics_2}

        OpenVPN dovoluje autentizovat své připojené členy použitím tzv. před-sdíleného klíče (pre-shared secret key), certifikátů nebo kombinací uživatelského jména a hesla. Když je použit ve více-klientské konfiguraci, pak OpenVPN umožňuje, aby server zpřístupňil autentizační certifikát zvlášť pro každého uživatele za použití podpisů a certifikační autority. Stejně jako OpenSSL využíva naplno šifrovací knihovny OpenSSL a protokol SSLv3/TLSv1. Obsahuje velmi mnoho bezpečnostních a kontrolních prvků.

        OpenVPN byl portován na spoustu systémů a architektur, je používán v mnoho zabudovaných (embedded) systémech. Jako příklad lze uvést DD-WRT, který obsahuje OpenVPN server. Další například pouze implementují protokol OpenVPN, jako například SoftEther VPN, který implementuje celou řadu VPN protokolů.

        Kolem OpenVPN vznikla silná komunita, je dostupný pro všechny známé operační systémy.

      \subsubsection{Síťová implementace}

        OpenVPN dokáže běžet přes User Datagram Protocol (UDP) nebo Transmission Control Protocol (TCP), multiplexujíce vytvořené SSL tunely na jediném TCP/UDP protu\cite{openvpn_tls_mode_options} (RFC 3948 pro UDP).\cite{openvpn_udp}

        Od verze 2.3.x dál OpenVPN plně podporuje IPv6 jako protokol virtuální sítě uvnitř tunelu a OpenVPN aplikace tak mohou navazovat spojení pomocí IPv6.\cite{openvpn_community_wiki}

        Dokáže pracovat skrze většinu proxy serverů (dokonce i HTTP proxy) a velmi dobře zvládá komunikaci i skrze překladače síťových adres (NAT). Obvykle nebývá blokován různými firewally.
        Server OpenVPN dokáže publikovat určité síťové nastavení svým klientům. Tyto zahrnují přiřazené IP adresy, příkay routování a další spojené s připojením.
        OpenVPN nabízí dva typy rozhraní pro vytvoření síťe pomocí univerzálího TUN/TAP ovladače. Dokáže vytvořit jak IP tunely OSI vrstvy 3 (TUN), tak Ethernet TAP (OSI vrstva 2), která dokáže přenášet jakýkoliv ethernetový provoz. Dále také umožnuje komprimovat proudově přenšená data.

      \subsubsection{Šifrování}
        OpenVPN používá k šifrování dat a kontrolního provozu knihovnu OpenSSL. Nechávý knihovnu OpenSSL dělat veškeré šífrovací i autentizační práci. OpenVPN tak může využít veškerých šifer implementovaných v balíčku knihovny OpenSSL.

        Může také využívat prvek paketové HMAC autentizace, aby tak obalil stávájící zabezpečení připojeni další vrstvou (tvůrce to označuje jako HMAC Firewall). Může také používat hardwarovou akceleraci, aby bylo dosaženo lepšího výkonu šifrování.\cite{openvpn_enc_1}\cite{openvpn_enc_2}

        Podpora PolarSSL je k dispozici od verze 2.3 a vyšší.\cite{openvpn_enc_3}

	  \subsubsection{Zabezpečení}
	  
		OpenVPN nabízí mnoho interních zabezpečovacích prvků. Dokáže použít šifrování až do šířky 256 bitů pomocí knihovny OpenSSL, přesto však někteří poskytovatelé nabízejí nižší šířku aby dosáhli rychlejšího spojení (s šifrováním je spojen výpočetní overhead).\cite{openvpn_sec}
		OpenVPN beží v uživatelském prostoru, nevyžaduje tedy zasahovat do IP stacku jádra. OpenVPN umí zanechat svá administrátorská práva, aby efektivně pracoval s nižšími oprávněními. Dále umí využít mlockall, aby předešel zápisu citlivých dat na disk v případě odložení (swapování), vstoupit do vězení typu chroot a aplikovat SELinux kontext po inicializaci.

		Používá vlastní protokol postavený na SSL a TLS místo toho, aby podporoval IPSec, L2TP nebo PPTP. OpenVPN nabízí podporu smart čipových karet pomocí PKCS\#11 k autentizaci šifrovacími klíči.
		
    \subsection{PPTP}

      \subsubsection{Popis}
       
		Point-to-Point tunelovací protokol (PPTP) je metoda pro implementování virtualních privátních sítí. PPTP používá kontrolné kanál přes TCP a GRE tunel obalující PPP pakety.

		Specifikace PPTP nepopisuje šifrování nebo autentizační prvky a spoléhá na to, že Point-to-Point protokol bude provozován skrze tunel, aby se zajistily bezpečnostní funkcionalita.
		Nejrozšířenější implementace PPTP je však dodávána s operačním systémem Microsoft Windows, které implementují různé úrovně autentizace a šifrování nativně jako standardní součást stacku PPTP ve Windows.
		Předpokládaný účel tohoto protokolu je poskytnout zabezpečení a vzdálený přístup na úrovni srovnatelný s jinými typickými VPN produkty.

		Specifikace PPTP byla publikována v červenci 1999 v RFC 2637\cite{pptp_basics} a byla vyvinuta sdružením společností Microsoft, Ascend Communications, 3Com a dalších. PPTP nebylo navrhnuto ani schváleno jako standard komisí pro technickou stránku internetu (Internet Engineering Task Force).
		
	  \subsubsection{Bezpečnost PPTP}
	  
		PPTP bylo podrobeno mnoha bezpečnostním analýzám a v tomto protokolu byly nalezeny závažné bezpečnostní problémy.
		Známe zranitelnosti souvisí s použitým ověřovacím protokolem PPP, s designem MPPE šifrování (Microsoft Point-to-Point Encryption) protokolu stejně jako s integrací autentizace mezi MPPE a PPP pro navázání klíče sezení (session key).\cite{pptp_unsec_1}\cite{pptp_unsec_2}
		
    \subsection{L2TP}
	
      \subsubsection{Popis}
		L2TP je protokol pro zabezpeční tunelu k přenosu IP provozu použitím PPP. L2TP zaobaluje PPP ve virtualních linkách, které běží přes IP, Frame Relay nebo jiné protokoly.
		L2TP používá PPP a MPPE k zašifrování spojení. Účel tohoto protokolu je umožnit provozu 2. vrstvy OSI modelu a koncovým bodům PPP ležet na jiných zařízeních spojených sítí s přepínáním paketů. S L2TP má uživatel spojení 2. vrstvy do přístupového koncetrátoru (access concentrator - LAC), což může být modem, ADSL DSLAM atd. a koncentrátor, který tuneluje individuální PPP rámce do (Network Access Server - NAS). Toto umožňuje vlastní zpracování PPP paketů oddělené od konců spojení 2. vrstvy. Z uživatelského pohledu není žádný funkční rozdíl mezi tím mít konce spojení 2. úrovně končíce v NASu přímo nebo při použití L2TP.

		Je užitečné používat L2TP jako kterýkoliv jiný tunelovací protokol s nebo bez šifrování. Standard L2TP říká, že nejbezpečnější formou, jak zašifrovat data, je použit L2TP přes IPsec (toto je výchozí pro klienta L2TP od Microsoftu), jelikož veškeré kontrolní či datové pakety pro určitý tunel se jeví jako homogenní UDP/IP pakety v IPsec systému.

		Vícelinkové PPP (Multilink PPP - MP) je podporováno, aby poskytlo MRRU (umožnění posílat pakety s plnou velikostí 1500 a větší) a zároveň přemostění (bridge) přes PPP linky (pomocí Bridge Control protokolu (BCP), který dovoluje posílat holé ethernetové rámce přes PPP linky. Touto cestou je možné nastavit přemostení bez použití ethernet přes IP (Ethernet over IP - EoIP). Most by měl mít administrátorsky nastavenou MAC adresu nebo ethernetu podobné rozhraní, jelikož PPP linky nemívají MAC adresy.

		L2TP obsauje PPP autentizaci a správu účtů pro každé L2TP spojení. Úplná autentizace a správa účtů každého spojení může být zprovozněna skrze RADIUS klient nebo lokálně.

		Jako ěifrovací metody jsou podporovány: MPPE 40bit RC4 a MPPE 128bit RC4.

		Provoz L2TP používá UDP protokol pro kontrolní i datové pakety. UDP port 1701 je použit pouze pro navázání spojení, dalši provoz již procházi skrze jakýkoliv UDP port (který nemusí být 1701). To znamená, že L2TP může být použit s většinou firewallů a routerů (dokonce i s NAT) povolením UDP provozu, aby byl routován skrze firewall nebo router.
        \cite{l2tp_basics}

\chapter{Bezpečnost}

\begin{conclusion}

    

\end{conclusion}

\bibliographystyle{csn690}
\bibliography{biblio}

\appendix

\chapter{Seznam použitých zkratek}
% \printglossaries
\begin{description}

    \item[TODO] TODO

\end{description}

\end{document}
